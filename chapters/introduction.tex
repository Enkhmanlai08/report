\section{Программууд судлах}
\subsection{RemotePC}
\footnote{RemotePC official site \url{https://www.remotedesktop.com}}
	\quad \quad RemotePC нь хэрэглэгчдэд интернэтэд холбогдсон дурын төхөөрөмж ашиглан хаанаас ч хамаагүй компьютер эсвэл сервертээ хандах, удирдах боломжийг олгодог алсаас хандалтын программ хангамжийн шийдэл юм. RemotePC-ийн талаарх зарим мэдээлэл энд байна:

1. Онцлогууд: RemotePC нь дараах функцүүдийг санал болгодог.

- Алсын компьютерт хандах: Хэрэглэгчид компьютер дээрээ бодитоор сууж байгаа мэт алсаас хандах боломжтой.
 
- Платформ хоорондын дэмжлэг: RemotePC нь Windows, Mac, Linux, iOS, Android зэрэг янз бүрийн үйлдлийн системүүдтэй нийцдэг.

- Файл дамжуулах: Хэрэглэгчид алсаас компьютер болон түүнд хандаж байгаа төхөөрөмж хооронд файл дамжуулах боломжтой.

- Алсын зайнаас хэвлэх: Хэрэглэгчид бичиг баримтыг алсаас дотоодын хэвлэгчээр хэвлэх боломжтой.

- Аюулгүй холболт: RemotePC нь TLS v 1.2/AES-256 шифрлэлт болон Хувийн түлхүүрийн баталгаажуулалтаар дамжуулан аюулгүй холболтыг баталгаажуулдаг.

2. Төлөвлөлт ба үнэ: RemotePC нь хувь хүн, жижиг бизнес, аж ахуйн нэгжүүдийн хэрэгцээнд тохирсон өөр өөрөөр санал болгодог. Үнэ нь ихэвчлэн хандахыг хүсэж буй компьютерын тоо болон шаардлагатай функцүүдээс хамаарч өөр өөр байдаг.

3. Суулгах ба тохируулах: Хэрэглэгчид ихэвчлэн алсаас хандахыг хүссэн компьютер болон тэдгээрт хандахын тулд ашиглах төхөөрөмждөө RemotePC программыг суулгах шаардлагатай болдог. Тохируулах процесс нь ихэвчлэн бүртгэл үүсгэх, RemotePC программ хангамжийг суулгах, алсаас хандах тохиргоог хийх зэрэг орно.

4. Хэрэглээ : 

- Алсын техникийн дэмжлэг: Мэдээллийн технологийн мэргэжилтнүүд алслагдсан компьютер дээрх асуудлыг биечлэн байхгүй ч олж засварлаж, засах боломжтой.

- Алсаас ажиллах: RemotePC нь ажилчдад гэрээсээ эсвэл аялж байхдаа ажлынхаа компьютерт хандах боломжийг олгож, алсаас ажиллах боломжийг олгодог.

- Хувийн хэрэглээ: Хувь хүмүүс RemotePC-ийг гэрийн компьютертоо хандах эсвэл гэр бүлийн гишүүд, найз нөхдөдөө техникийн туслалцаа үзүүлэх зорилгоор ашиглаж болно.

5. Аюулгүй байдал: RemotePC нь алсаас найдвартай холболтыг баталгаажуулахын тулд шифрлэлт болон баталгаажуулалтын протоколуудыг хэрэгжүүлснээр аюулгүй байдлыг хангадаг. Энэ нь нууц мэдээллийг хамгаалах, алсын компьютерт зөвшөөрөлгүй нэвтрэхээс сэргийлэхэд тусална.

Ерөнхийдөө RemotePC нь компьютерт алсаас хандах, удирдах найдвартай, найдвартай арга хайж буй хувь хүмүүс, бизнес эрхлэгчид болон мэдээллийн технологийн мэргэжилтнүүдэд зориулсан алсаас хандалтын олон талт шийдэл юм.
\pagebreak

\subsection{TeamViewer}
\footnote{TeamViewer official site \url{https://www.teamviewer.com/en/}}
	\quad \quad TeamViewer нь хэрэглэгчдэд дэлхийн хаанаас ч алсаас төхөөрөмжид хандах, удирдах боломжийг олгодог алдартай алсаас хандах болон алсын удирдлагатай программ хангамжийн шийдэл юм.

1. Онцлогууд: 

	- Зайнаас хандалт: Хэрэглэгчид компьютер, сервер эсвэл хөдөлгөөнт төхөөрөмжүүдэд бие махбодтой байгаа мэт алсаас хандаж, удирдах боломж.

	- Платформ хоорондын дэмжлэг: TeamViewer нь Windows, macOS, Linux, iOS, Android зэрэг янз бүрийн үйлдлийн системүүдтэй нийцдэг.

	- Файл дамжуулах: Хэрэглэгчид дотоод болон алсын төхөөрөмжүүдийн хооронд файлуудыг аюулгүйгээр дамжуулах боломжтой.

	- Хамтран ажиллах хэрэгслүүд: TeamViewer нь дэлгэц хуваалцах, самбар зэрэг онлайн уулзалт, танилцуулга, хамтран ажиллах боломжуудыг агуулдаг.

	- Аюулгүй холболт: TeamViewer нь төгсгөл хоорондын шифрлэлт, олон хүчин зүйлийн баталгаажуулалт, хандалтын хяналтаар дамжуулан аюулгүй холболтыг баталгаажуулдаг.

2. Төлөвлөлт ба үнэ: TeamViewer нь хувийн, бизнесийн болон байгууллагын хэрэглэгчдэд зориулсан өөр өөр санал болгодог. Үнэ нь ихэвчлэн хэрэглэгчид болон төхөөрөмжүүдийн тоо, мөн төлөвлөгөөнд тусгагдсан онцлогоос хамаарч өөр өөр байдаг. Захиалгат болон нэг удаагийн худалдан авалтын сонголтууд байдаг.

3. Суулгах ба тохируулах: Хэрэглэгчид TeamViewer-ийг ашиглахын тулд алсаас хандахыг хүссэн төхөөрөмж болон түүнд хандахдаа ашиглах төхөөрөмждөө хоёуланд нь TeamViewer программ хангамжийг татаж аваад суулгах шаардлагатай. Тохируулах процесс нь TeamViewer бүртгэл үүсгэх, программ хангамжийг суулгах, алсын зайнаас хандах тохиргоог тохируулах явдал юм.

4. Хэрэглэх тохиолдлууд: TeamViewer-ийг янз бүрийн зорилгоор ашигладаг, үүнд:

- Мэдээллийн технологийн алсаас дэмжлэг үзүүлэх: Мэдээллийн технологийн мэргэжилтнүүд алслагдсан компьютер, сервер эсвэл хөдөлгөөнт төхөөрөмж дээрх асуудлыг засах боломжтой.

- Алсын зайнаас ажиллах: TeamViewer нь ажилчдад ажлын компьютертоо хандах эсвэл хамтран ажиллагсадтайгаа алсаас хамтран ажиллах боломжийг олгож, алсаас ажиллах, виртуал уулзалт хийх боломжийг олгодог.

- Хувийн хэрэглээ: Хувь хүмүүс TeamViewer-ийг ашиглан гэрийн компьютертоо хандах, гэр бүл, найз нөхдөдөө техникийн асуудлаа шийдвэрлэхэд нь туслах эсвэл аялж байхдаа төхөөрөмждөө хандах боломжтой.

5. Аюулгүй байдал: TeamViewer нь нууц мэдээллийг хамгаалах, зөвшөөрөлгүй хандалтаас сэргийлэхийн тулд салбарын стандартын шифрлэлтийн протокол, аюулгүй байдлын арга хэмжээг хэрэгжүүлснээр аюулгүй байдлыг эрхэмлэдэг. Үүнд төгсгөл хоорондын шифрлэлт, холболтын нууц үг, хоёр хүчин зүйлийн баталгаажуулалт орно.

Ерөнхийдөө TeamViewer нь төхөөрөмжид алсаас хандах, удирдах, онлайнаар хамтран ажиллах, алсаас дэмжлэг үзүүлэх найдвартай арга хайж буй хувь хүмүүс, бизнес эрхлэгчид болон мэдээллийн технологийн мэргэжилтнүүдэд тохиромжтой алсаас нэвтрэх олон талын шийдэл юм.
			
\pagebreak

\subsection{Zoho Assist}
\footnote{Zoho Assist official site \url{https://www.zoho.com/assist/remote-access-control.html}}
	\quad \quad Zoho Assist нь Zoho корпорацын хөгжүүлсэн алсын удирдлага, алсаас хандах программ хангамж юм. Энэ нь мэдээллийн технологийн мэргэжилтнүүд болон туслах багуудад компьютер, сервер, хөдөлгөөнт төхөөрөмжүүдэд алсаас хандах, алдааг олж засварлах хэрэгслүүдээр хангадаг.

1. Онцлогууд: 

- Алсаас хандах: Техникчид интернэт холболттой хаанаас ч алсын компьютер, сервер, хөдөлгөөнт төхөөрөмжид хандаж, удирдах боломжтой.

- Дэлгэц хуваалцах: Хэрэглэгчид үзүүлэн үзүүлэх, танилцуулга хийх эсвэл хамтран ажиллах зорилгоор дэлгэцээ алсын хэрэглэгчидтэй хуваалцах боломжтой.

- Файл дамжуулах: Zoho Assist нь техникч болон алсын хэрэглэгчийн төхөөрөмжүүдийн хооронд аюулгүй файл дамжуулах боломжийг олгодог.

- Хяналтгүй хандалт: Техникчид байнгын засвар үйлчилгээ эсвэл алдааг олж засварлах зорилгоор алсын компьютерт хяналтгүй хандалтыг тохируулах боломжтой.

- Олон платформын дэмжлэг: Zoho Assist нь Windows, macOS, Linux, iOS, Android зэрэг янз бүрийн үйлдлийн системүүдтэй нийцдэг.

- Хадгалах: Хэрэглэгчид баримтжуулалт эсвэл сургалтын зорилгоор алсаас дэмжлэг үзүүлэх холболтыг хадгалах боломжтой.

- Захиалга: Zoho Assist нь компанийн брэнд, лого бүхий дэмжлэгийн порталыг өөрчлөх боломжийг олгодог.

2. Төлөвлөлт ба үнэ: Zoho Assist нь хувь хүмүүс, жижиг бизнесүүд, аж ахуйн нэгжүүдийн хэрэгцээнд тохирсон өөр өөр үнийг санал болгодог. Үнэ нь ихэвчлэн шаардлагатай техникийн ажилтны тоо болон зэрэгцээ холболт зэргээс хамаарч өөр өөр байдаг.

3. Суулгах ба тохируулах: Zoho Assist-ийг ашиглахын тулд техникч болон алсын хэрэглэгч хоёулаа Zoho Assist клиент программ хангамжийг тус тусын төхөөрөмж дээрээ суулгах шаардлагатай. Тохируулах процесс нь Zoho данс үүсгэх, программ хангамжийг татаж авах, алсаас хандах тохиргоог хийх зэрэг орно.

4. Хэрэглээ: Zoho Assist-ийг янз бүрийн зорилгоор ашигладаг, үүнд:

- Мэдээллийн технологийн алсаас дэмжлэг үзүүлэх: Мэдээллийн технологийн мэргэжилтнүүд хэрэглэгчийн компьютер, сервер, гар утасны төхөөрөмж дээрх техникийн асуудлыг алсаас олж засварлаж, шийдвэрлэх боломжтой.

- Тусламжийн ширээний дэмжлэг: Тусламжийн баг нь техникийн асуудалтай тулгарсан үйлчлүүлэгч эсвэл ажилчдад алсаас тусламж үзүүлэх боломжтой.

- Сургалт ба хамтын ажиллагаа: Zoho Assist-ийг оролцогчдод дэлгэцээ хуваалцах, алсаас харилцах боломжийг олгох замаар онлайн сургалт, танилцуулга, хамтын ажилд ашиглах боломжтой.

5. Аюулгүй байдал: Zoho Assist нь алсын холболт болон өгөгдлийг хамгаалахын тулд шифрлэлтийн протокол, аюулгүй байдлын арга хэмжээг хэрэгжүүлснээр аюулгүй байдлыг нэн тэргүүнд тавьдаг. Үүнд өгөгдөл дамжуулахад зориулсан AES 256 битийн шифрлэлт, аюулгүй харилцааны TLS 1.2 протокол орно.

Ерөнхийдөө Zoho Assist нь алсаас тусламж үзүүлэх, техникийн асуудлыг шийдвэрлэхэд хялбар, аюулгүй аргыг хайж буй мэдээллийн технологийн мэргэжилтнүүд, туслах баг, бизнесүүдэд тохиромжтой алсаас дэмжлэг үзүүлэх, алсаас нэвтрэх цогц шийдэл юм.

\pagebreak

\subsection{TSplus}
\footnote{TSplus official site \url{https://tsplus.net/}}
	\quad \quad TSplus нь бизнес эрхлэгчдэд хаанаас ч, ямар ч төхөөрөмжөөс өөрийн application болон ширээний компьютерт аюулгүй хандах, удирдах боломжийг олгодог алсын ширээний компьютер болон программ хангамжийн шийдэл юм.

1. Онцлогууд: 

- Алсаас хандах: Хэрэглэгчид интернэт холболттой дурын төхөөрөмжөөс өөрийн ширээний компьютер эсвэл виртуалчлагдсан программдаа алсаас хандаж, удирдах боломжтой.

- Application нийтлэх: TSplus нь бизнесүүдэд алсын хэрэглэгчдэд программ нийтлэх боломжийг олгож, ширээний компьютерт бүрэн хандалт өгөхгүйгээр тодорхой программ хангамжид хандах боломжийг олгодог.

- HTML5 клиент: TSplus нь HTML5 веб клиентээр хангадаг бөгөөд энэ нь хэрэглэгчдэд нэмэлт программ хангамж суулгах шаардлагагүйгээр веб хөтчөөс шууд программууд болон ширээний компьютерт хандах боломжийг олгодог.

- Ачаалал тэнцвэржүүлэх: TSplus нь хэрэглэгчийн холболтыг олон серверт түгээх ачааллыг тэнцвэржүүлэх чадварыг багтаасан бөгөөд оновчтой гүйцэтгэл, найдвартай байдлыг хангадаг.

- Алсын зайнаас хэвлэх: Хэрэглэгчид өөрийн алсаас компьютер эсвэл программын холболтоос дотоод принтерт бичиг баримтыг алсаас хэвлэх боломжтой.

- Аюулгүй байдал: TSplus нь алсын холболтоос болон өгөгдлийг хамгаалахын тулд шифрлэлтийн протокол, хандалтын хяналт, баталгаажуулалтын механизмыг хэрэгжүүлснээр аюулгүй байдлыг нэн тэргүүнд тавьдаг.

2. Лицензүүд: TSplus нь үндсэн, гар утас, аж ахуйн нэгжийн хувилбар зэрэг бүх төрлийн бизнесийн хэрэгцээнд нийцсэн өөр хувилбаруудыг санал болгодог. Лиценз нь хэрэглэгчийн тоо эсвэл шаардлагатай зэрэгцээ холболтоос хамааран өөр өөр байдаг.

3. Суулгах ба тохируулах: TSplus-ийг тохируулах нь хэрэглэгчдийн алсаас хандах шаардлагатай программууд эсвэл компьютер байршуулсан сервер дээр TSplus программ хангамжийг суулгах. Тохируулах процесс нь хэрэглэгчийн зөвшөөрөл, программыг нийтлэх, аюулгүй байдлын тохиргоог багтаана. 

4. Хэрэглээ: 

- Алсын ажил: Бизнесүүд TSplus-ийг ашиглан ажилчдаа ажлын ширээний компьютер болон программдаа алсаас хандах боломжийг олгож, алсаас ажиллах болон зайнаас ажиллах боломжийг хөнгөвчлөх боломжтой.

- Хүртээмж: TSplus нь бизнесүүдэд программуудыг өөр өөр байршилд төвлөрсөн байдлаар удирдах, хэрэглэгчдэд хүргэх боломжийг олгож, мэдээллийн технологийн ачаалал, нарийн төвөгтэй байдлыг багасгадаг.

- Алсын дэмжлэг: Мэдээллийн технологийн мэргэжилтнүүд TSplus-ийг ашиглан алсаас тусламж үзүүлж, хэрэглэгчийн ширээний компьютер эсвэл программ дээрх техникийн асуудлыг шийдвэрлэх боломжтой.

5. Дэмжлэг ба шинэчлэлт: TSplus нь хэрэглэгчид хамгийн сүүлийн үеийн функцүүд болон аюулгүй байдлын хангамжийг ашиглах боломжийг хангах үүднээс техникийн дэмжлэг, программ хангамжийн шинэчлэлтүүдийг санал болгодог.

Ерөнхийдөө TSplus нь алсаас нэвтрэх орчинд бүтээмж, уян хатан байдал, аюулгүй байдлыг сайжруулахыг эрэлхийлж буй бизнесүүдэд тохиромжтой, алсын ширээний компьютер болон программ хангамжийн найдвартай шийдэл юм.
\pagebreak

\section{Шийдлүүдийг боловсруулах}
\subsection{Харьцуулалт}
	\quad \quad 
	RemotePC, TeamViewer, Zoho Assist гэсэн гурван программын давуу болон сул талуудыг харьцуулж, аль нь их сургууль болон ажлын орчинд илүү тохиромжтой болохыг үнэлье.

	1. RemotePC:
	
	Давуу тал:
	
	- Хувь хүн, аж ахуйн нэгжүүдэд боломжийн үнийн санал.
	
	- Энгийн бөгөөд хэрэглэгчдэд хялбар интерфейс.
	
	- Файл дамжуулах чадвар сайтай.
	
	- Шифрлэлт, баталгаажуулалт бүхий аюулгүй холболт.
	
	- Хувийн хэрэглээ болон жижиг бизнест тохиромжтой.
	
	Сул тал:
	
	- Бусад шийдлүүдтэй харьцуулахад зарим дэвшилтэт боломжууд дутмаг.
	
	- Хязгаарлагдмал хамтын ажиллагааны хэрэгсэл.
	
	- Томоохон аж ахуйн нэгжүүдийн хувьд тийм ч хангалттай биш.

	2. TeamViewer:

	Давуу тал:
	
	- Алсаас компьютерт хандах, дэлгэц хуваалцах, файл дамжуулах зэрэг өргөн боломжууд.
	
	- Олон платформын дэмжлэг.
	
	- Онлайн уулзалт, танилцуулга хийхэд зориулсан маш сайн хэрэгсэл.
	
	- Шифрлэлт, баталгаажуулалт бүхий аюулгүй байдлын хүчтэй шийдэл.
	
	- Хувийн болон бизнесийн хэрэглээ, тэр дундаа томоохон аж ахуйн нэгжүүдэд тохиромжтой.
	
	Сул тал:

	- Үнийн хувьд бусад шийдлүүдтэй харьцуулахад харьцангуй өндөр.

	- Хэрэглэгчид хааяа холболтын асуудлын гомдол хэлдэг.

	3. Zoho Assist:

Давуу тал:

- Алсаас хандах, дэлгэц хуваалцах, хараа хяналтгүй хандалт зэрэг алсаас туслах иж бүрэн боломжууд.

- Бизнесийн хувьд боломжийн үнийн санал.

- Бүтээмжийг нэмэгдүүлэхийн тулд бусад Zoho бүтээгдэхүүнтэй нэгтгэж болдог.

- Шифрлэлт, хандалтын хяналт бүхий аюулгүй байдлын сайн арга зохицуулалт.

- Бүх төрлийн бизнест тохиромжтой.

Сул тал:

- Бусад шийдлүүд шиг хамтын ажилд зориулсан өргөн боломжууд байхгүй.

- Бусад алсаас хандах хэрэгслүүдтэй харьцуулахад хязгаарлагдмал тохируулгатай.

4. TSplus 

Давуу тал:

- Хэрэглэгчид өөрсдийн ширээний компьютер эсвэл application-нд алсаас хандах боломжийг олгож, уян хатан байдлыг нэмэгдүүлнэ.

- Application-нд сонгон хандахыг зөвшөөрч, аюулгүй байдлыг сайжруулна.

- Өргөн хүртээмжтэй байх үүднээс янз бүрийн үйлдлийн системүүд дээр ажилладаг.

-  Илүү сайн ажиллахын тулд хэрэглэгчийн холболтыг түгээдэг.

- Тохиромжтой болгох үүднээс алсаас хэвлэх боломжийг олгоно.

- Өгөгдлийн хамгаалалтад зориулсан шифрлэлт болон хандалтын хяналтыг багтаасан.

Сул тал:

- Том хэмжээний тохиргоонд анхны хөрөнгө оруулалт өндөр байж магадгүй.

- Тохиргоо нь техникийн болон төвөгтэй байж магадгүй.

- Шинэчлэлт болон хэрэглэгчтэй ажиллахад байнгын удирдлага шаардлагатай.

- Өргөтгөхөд нэмэлт төлөвлөлт шаардлагатай байж магадгүй.

- Тогтвортой сүлжээний холболтод тулгуурлан саадгүй нэвтрэх боломжтой.
	\pagebreak	

\subsection{Шийдэл}
	\quad \quad 
	Их сургууль болон ажлын орчинд аль нь илүү дээр вэ?

- Зардлын нөлөө чухал бол RemotePC нь боломжийн үнэтэй учир тохиромжтой байж болно.

- Хэрэв хамтын ажиллагааны дэвшилтэт хэрэгсэл, өргөн боломжууд, ялангуяа томоохон аж ахуйн нэгжүүдэд шаардлагатай бол TeamViewer илүү тохиромжтой.

- Өөр программуудтай нэгтгэх чухал бол Zoho Assist илүү тохиромжтой.

- Миний бодлоор ажлын болон их сургуулийн орчинд аль алинд нь энэхүү программ хангамж шаардлага хэрэгцээ байгаа гэж үзэж байна.

-Жишээ 1. Хааяа хувийн компьютергүй их сургууль дээрээ очих өдөр элбэг байдаг. Их сургуулийн лабораторийн компьютер дээр ажилласан даалгавраа өөрийн хувийн компьютер лүүгээ авах гэж заавал лабораторийн компьютер дээр Teams, Facebook гэх мэт программ ашиглан хувийн хаягаараа дамжуулж авдаг. Энэ нь маш эрсдэлтэй юм. Хувийн хаягаа өөр хүнд алдахаас эхлээд нэр төрдөө халдуулах ч үр дагавар гарж болзошгүй.

-Жишээ 2. Ажлын төрлөөс хамаараад өөр өөрийн гэсэн онцлогууд байж болох ч одоо үед ихэнх ажлын газрууд байгууллагын нууцлалын бодлогын дагуу мэдээллийн аюулгүй байдлыг ханган ажлын компьютерыг ашиглах нь элбэг. Хэрвээ би 24/7 хараа хяналттай байх ёстой ажлын байранд ажилладаг гэж үзье(NOC engineer etc). Оффисоос гадуур явж байх үед гэнэт гэмтэл саатал гарах үед асуудал үүснэ. Ажлын газрын суурин компьютер дээр надад шаардлагатай файл байж болзошгүй.
	\pagebreak	   
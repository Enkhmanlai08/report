\section{Сервер}
Сервер гэдэг нь сүлжээн дэх бусад компьютер эсвэл төхөөрөмжүүдэд үйлчилгээ, нөөцөөр хангадаг компьютер эсвэл систем юм. Энэ нь үндсэндээ хүсэлтийн дагуу бусад компьютер, төхөөрөмж, хэрэглэгчид байж болох мэдээлэл, файл, программ эсвэл нөөцийг үйлчлүүлэгчдэд дамжуулдаг. Серверүүд нь веб сайт байршуулах, файл хадгалах, удирдах, сүлжээний урсгалыг удирдах, программуудыг ажиллуулах, мэдээллийн санд хандах хандалтыг хангах зэрэг төрөл бүрийн ажлуудыг гүйцэтгэхэд зориулагдсан.

Серверүүд нь тодорхой функцээс хамааран өөр өөр төрлөөр ирдэг. Зарим нийтлэг төрлийн серверүүд нь:

1. Веб серверүүд: Эдгээр серверүүд нь веб хуудаснуудыг байршуулж, веб хөтчөөр дамжуулан хүссэн хэрэглэгчдэд веб хуудаснуудаар үйлчилдэг.

2. Файл серверүүд: Файл серверүүд нь сүлжээнд байгаа олон хэрэглэгч эсвэл төхөөрөмжид хандаж, хуваалцаж болох файлуудыг хадгалж, удирддаг.

3. Өгөгдлийн сангийн серверүүд: Өгөгдлийн сангийн серверүүд нь өгөгдлийн санг хадгалж, удирдаж, үйлчлүүлэгчдэд тэдгээрт хадгалагдсан өгөгдлийг сэргээх, удирдах боломжийг олгодог.

4. Мэйл серверүүд: Мэйл серверүүд нь байгууллага эсвэл сүлжээн дэх хэрэглэгчдэд мэйл мессеж илгээх, хүлээн авах, хадгалах үйл ажиллагааг зохицуулдаг.

5. Хэрэглээний серверүүд: Хэрэглээний серверүүд нь ихэвчлэн тархсан тооцооллын орчинд программуудыг ажиллуулах, удирдахад шаардлагатай дэд бүтэц, үйлчилгээгээр хангадаг.

6. DNS серверүүд: DNS (Домэйн Нэрийн Систем) серверүүд нь домэйн нэрийг IP хаяг руу хөрвүүлж, хүний унших боломжтой хаягийг ашиглан веб сайт болон бусад эх сурвалжид хандах боломжийг хэрэглэгчдэд олгодог.

7. Proxy серверүүд: Proxy серверүүд нь үйлчлүүлэгч болон бусад серверүүдийн хооронд зуучлагчийн үүрэг гүйцэтгэж, гүйцэтгэл, аюулгүй байдал эсвэл нууцлалыг сайжруулахын тулд хүсэлт, хариултыг дамжуулдаг.

Эдгээр нь сүлжээний орчинд тус бүр нь тодорхой зорилгоор үйлчилдэг олон төрлийн серверүүдийн цөөн хэдэн жишээ юм. Серверүүд нь ихэвчлэн зориулалтын функцээ үр дүнтэй, найдвартай гүйцэтгэх зориулалттай тусгай программ хангамжийг ажиллуулдаг.

Миний хувьд "Mobicom Corporation LLC" Software Developer-оор ажиллаж байгаа учир ажлын газраас зөвшөөрөлтэй, түр ашиглах нөхцөлтэйгөөр test server ашиглах эрхтэй болсон.

Энэхүү тест серверийг файл сервер болгон хөгжүүлсэн бөгөөд байгууллагын нууцлалын бодлогоос хамааран тайлагнах зөвшөөрөл хараахан байхгүй болно.
\pagebreak
\section{Клиент}
Компьютерын сүлжээний хүрээнд үйлчлүүлэгч гэдэг нь серверээс үйлчилгээ, эх сурвалжийг хүсэх компьютерын программ эсвэл төхөөрөмжийг хэлнэ. Үйлчлүүлэгчид серверээс өгсөн өгөгдөл, үйлчилгээ эсвэл программд хандахын тулд серверүүдтэй харилцах харилцааг эхлүүлдэг. Үйлчлүүлэгчид нь компьютер, ухаалаг гар утас, таблет эсвэл бусад сүлжээнд холбогдсон төхөөрөмж дээр ажилладаг программ хангамжийн программууд байж болно.

Үйлчлүүлэгч болон серверийн хооронд харилцаа холбоо, өгөгдөл дамжуулахад туслах хэд хэдэн бүрэлдэхүүн хэсэг байдаг. Үндсэн бүрэлдэхүүн хэсэг нь үйлчлүүлэгч болон серверийн хооронд өгөгдөл дамжуулах боломжийг олгодог чиглүүлэгч, унтраалга, кабель зэрэг янз бүрийн сүлжээний төхөөрөмжүүдийг багтаасан сүлжээ өөрөө юм.

1. Сүлжээний дэд бүтэц: Үүнд өгөгдөл дамжуулах физик сүлжээг бүрдүүлдэг чиглүүлэгч, унтраалга, кабель зэрэг техник хангамжийн төхөөрөмжүүд орно.

2. Протоколууд: Үйлчлүүлэгч болон сервер хоорондын харилцаа холбоо нь ихэвчлэн TCP/IP (Transmission Control Protocol/Internet Protocol), HTTP (Hypertext Transfer Protocol), FTP (File Transfer Protocol) болон бусад сүлжээний протоколд тулгуурладаг. Эдгээр протоколууд нь өгөгдөл дамжуулах дүрэм, дэг журам тодорхойлж, өөр өөр системүүдийн харилцан үйлчлэлийг хангадаг.

3. Галт хана ба хамгаалалтын арга: Сүлжээ болон түүний хөрөнгийг зөвшөөрөлгүй хандалт, хортой халдлага, өгөгдөл зөрчлөөс хамгаалахын тулд галт хана болон бусад хамгаалалтын арга хэмжээг хэрэгжүүлж болно.

4. Middleware: Зарим тохиолдолд үйлчлүүлэгч болон сервер хоорондын харилцааг хөнгөвчлөх зорилгоор дунд программыг ашиглаж болно. Middleware нь мессежний дараалал, гүйлгээний боловсруулалт, алсын зайн процедурын дуудлага (RPCs) зэрэг үйлчилгээгээр хангадаг программ хангамж бөгөөд үйлчлүүлэгч болон серверүүдтэй үр дүнтэй харилцах боломжийг олгодог.

5. Интернэт: Интернэтээр харилцахын тулд үйлчлүүлэгчид болон серверүүд хоорондоо холбогдсон төхөөрөмжүүдийн өргөн сүлжээгээр өгөгдөл дамжуулахын тулд чиглүүлэгч, унтраалга, интернэт үйлчилгээ үзүүлэгч (ISP) зэрэг интернэтийн дэд бүтцийн янз бүрийн бүрэлдэхүүн хэсгүүдэд тулгуурладаг.

Ерөнхийдөө эдгээр бүрэлдэхүүн хэсгүүд нь үйлчлүүлэгчид болон серверүүдийн хооронд харилцааны сувгийг бий болгохын тулд хамтран ажиллаж, компьютерын сүлжээгээр өгөгдөл, үйлчилгээ солилцох боломжийг олгодог.
\pagebreak